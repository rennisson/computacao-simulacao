\documentclass{article}
\usepackage{graphicx} % Required for inserting images
\usepackage{amsmath}
\usepackage{caption}
\usepackage[leftcaption]{sidecap}
\usepackage{afterpage}
\usepackage{titling}

\predate{}
\postdate{}
\graphicspath{ {./imagens/} }
\usepackage{multicol}
\date{}
\title{\vspace{-3.0cm}EP 05 - MAP2212}
\author{Gustavo Nunes - 13685534 \\Rennisson Davi Alves - 13687175}

\begin{document}

    \maketitle
        \section{O Programa}
        O objetivo é encontrar o e-valor da hipotese de Hardy-Weinberg no modelo da trinomial de Dirichlet.
        Para isso, usaremos como base o código desenvolvido no exercicio programa 5.

        De início, o vetor (1, 1, 1) foi selecionado como valor da priori. Outros valores e arrays foram gerados
        através da função Dirichlet da biblioteca SciPy. Na Dirichlet, o parâmetro usado é gerado aleatóriamente
        na função "main", assim como no EP05. Para melhora de performance, o Numpy foi utilizado, a fim de facilitar
        as operações com arrays.

        Estão abaixo os resultados de alguns testes realizados:

        \begin{table}[h]
            \centering
            \begin{tabular}{|c|c|c|c|c|}
                $x_1$ & $x_3$ & Ev & Precisão & Tempo de execução\\
                \hline
                1 & 4 & 0.039 & 1 & 1.0823 \\
                1 & 8 & 0.485 & 1 & 1.0469 \\
                1 & 8 & 0.485 & 1 & 1.0469 \\
                1 & 11 & 0.95* & 1 & 1.1099 \\
                1 & 15 & 0.665 & 0.98 & 1.1323 \\
                5 & 0 & 0.015 & 1 & 1.0471 \\
                5 & 1 & 0.094 & 1 & 1.1513 \\
                5 & 3 & 0.606 & 1 & 1.0913 \\
                5 & 5 & 1* & 1 & 1.1083 \\
                5 & 8 & 0.402 & 1 & 1.0471 \\
                9 & 0 & 0.193 & 0.95 & 1.0781 \\
                9 & 2 & 0.993* & 1 & 1.0768 \\
                9 & 4 & 0.497 & 1 & 1.0950 \\
                9 & 6 & 0.065 & 0.95 & 1.0939 \\
            \end{tabular}
        \end{table}

        Como base de comparação, utilizamos a tabela de valores presente em Evidence and Credibility:
        Full Bayesian Significance Test for Precise Hypotheses (Carlos Alberto de Bragança Pereira and Julio Michael Stern), que já nos mostra
        os valores que são esperados como resultado para cada conjunto de X.

        Para precisão, foram comparados os valores gerados entre as priori (1, 1, 1) e (0, 0, 0). Para tal, foram calculados os valores
        do conjunto X inputado no programa, utilizando a priori (1, 1, 1) e, logo após, utilizando a priori (0, 0, 0). No fim, foi calculada
        a razão entre os dois resultados. Como podemos observar, todos os valores tiveram precisão acima de 95\%. Isso significa que ambos os
        resultados originados da priori (1, 1, 1) e priori (0, 0, 0), chegaram em valores tão proximos quanto fossem possíveis,
        minimizando ao máximo o erro da função.


        \section{Conclusão}

        De acordo com os testes realizados, os resultados obtidos são muito satisfatórios se comparados
        com a tabela mencionada na seção anterior (Evidence and Credibility: Full Bayesian Significance Test for Precise Hypotheses).
        Vemos que todos os valores utilizados como input retornaram os valores esperados com precisão acima de 95\%.
        Tal feito denota o caráter preciso e valioso da avaliação da hipótese nula através do teste de Hardy-Weinberg.

        Com base nestes resultados, é válido dizer que a hipótese nula é aceita nos casos marcados com * na tabela da seção anterior,
        já que seus e-valores estão acima de 0.95.

        Além disso, a função apresentou tempos de execução bem satisfatórios, permitindo quantidades significativas de testes
        em tempo bastante enxutos, gastando poucos recursos computacionais.

        Vale lembrar que para gerar todos os arrays necessários, foi utilizado o NUSP
        13685534 como seed. Desse modo, fica mais fácil reproduzir os testes e verificar
        seus resultados.


\end{document}
