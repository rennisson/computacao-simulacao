\documentclass{article}
\usepackage{graphicx} % Required for inserting images
\usepackage{amsmath}
\usepackage{caption}
\usepackage[leftcaption]{sidecap}
\usepackage{afterpage}
\usepackage{titling}

\predate{}
\postdate{}
\graphicspath{ {./imagens/} }
\usepackage{multicol}
\date{}
\title{\vspace{-2.0cm}EP 02 - MAP2212}
\author{Gustavo Nunes / Rennisson Davi Alves}

\begin{document}
		\maketitle
	
	\section{O Programa}
	A ideia do programa é, a partir dos vetores $x$ e $y$ naturais de tamanho 3, gerados aleatóriamente. Dados $x$ e $y$, são gerados N $\{\theta \in R_3^+ | \theta^{'}=1 \}$. A densidade posterior de $\theta$ $f(\theta|x, y)$segue uma distribuição Dirichlet de parâmetro $\alpha = x + y$, e é assim que serão gerados os valores. 
	
	Definimos o conjunto de corte $T(v) = \{\theta \in \Theta | f(\theta | x, y) \leq v\}, v\geq0$. Daí, definimos a função verdade $W(v) = \int_{T(v)}f(\theta|x, y)d\theta$. 
	
	A ideia do programa é aproximar por métodos estocásticos $U(v)$, que serve como aproximação para $W(v)$ com erro menor que $0.05\%$. 
	
	Para isso, definimos k pontos de corte $0 = v_0 < v_1 < ... < v_k = sup f(\theta)$. Daí, geramos N pontos $\theta_1, \theta_2, ..., \theta_N$ de acordo com a Dirichlet, e simulamos $U(v)$ usando a fração dos pontos $\theta_k | f(\theta_k)  \leq v $. 
	
	\section{Escolha de N}
	
	Para a escolha do N empírico, usamos um N = $10^{6}$ suficientemente grande como referência para testes, com $\bar{U}(v)$ sendo gerado usando esse N. Também usamos $k = 10^5$(?), mais que o mínimo suficiente para gerar U na margem de erro definida. 
	
	Usamos uma função teste que gera 100 valores para $U(v)$, e devolve quantos estão dentro da margem de erro estabelecida, o valor i na tabela. Temos assim alguns resultados: 
 
	\begin{table}[htbp]
		\begin{tabular}{|c|c|c|c|c|}
			v & U(v) & Erro (\%) & Tempo (s)\\
			\hline
			0 & 0 & 0 & 0\\	
			4 & 0.1238 & 0.012 & 0.0469\\
			8 & 0.2596 & 0.026 & 0.0625\\
                12 & 0.4017 & 0.040 & 0.0937\\
                16 & 0.5481 & 0.054 & 0.1094\\
                20 & 0.6976 & 0.070 & 0.1562\\
                24 & 0.8489 & 0.085 & 0.1719\\
                28 & 1 & 0.1 & 0.2187\\
		\end{tabular}
	\end{table}
 

    \section{Testes e conclusão}
    
    De acordo com os testes realizados e mostrados na tabela da seção anterioro, os resultados obtidos são extremamente satisfatórios para o objetivo do que foi proposto. Para cada corte v, o algoritmo foi capaz de se aproximar do valor real da massa da função W(v) com precisão acima de 99.95\%.
    
    Vale lembrar que para gerar todos os arrays necessários, foi utilizado o NUSP 13685534 como seed. Desse modo, fica mais fácil reproduzir os testes e verificar seus resultados.
    
    Nao obstante, os tempos de execução de cada experimento foram excelentes, não ultrapassando o tempo de 1 segundo. Claro que o valor maximo de corte para analise da função é relativamente pequeno (27.93 para ser mais exato), tornando o número de comparações também pequeno. Mas como o algoritmo foi implementado utilizando ndarrays, mesmo se o valor maximo da função fosse muito maior, o programa não teria muito problema em executá-lo.
\end{document}