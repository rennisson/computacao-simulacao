\documentclass{article}
\usepackage{graphicx} % Required for inserting images
\usepackage{amsmath}
\usepackage{caption}
\usepackage[leftcaption]{sidecap}
\usepackage{afterpage}
\usepackage{titling}
\usepackage{algorithm}
\usepackage{algpseudocode}

\predate{}
\postdate{}
\graphicspath{ {./imagens/} }
\usepackage{multicol}
\date{}
\title{\vspace{-3.0cm}EP 02 - MAP2212}
\author{Gustavo Nunes }

\begin{document}
	
	\maketitle
	
	\section{O Programa}
	O programa é o mesmo do EP04, porém com uma mudança no gerador de números de acordo com a distribuição Diricihlet, sem usar o método rvs da função Dirichlet da biblioteca Scipy. 
	
	A ideia para o gerador é usar o algorítmo de Metropolis-Hastings, que para o nosso caso pode ser escrito em pseudocódigo: 
\begin{algorithm}
	\caption{Algorítmo de Metropolis-Hastings para gerador de números aleatórios de acordo com a distribuição Diricihlet}\label{}
	\begin{algorithmic}
		\State Theta1 = ponto arbitrário 
		\State f = array de resultados 
		\While{$c \leq N$}
		\State Theta2 gerado de Normal Multivariada de média Theta1
		\State Alpha = Dirichlet(Theta2) / Dirichlet(Theta1)
		\If{alpha $\geq$ 1} 
			\State f[c] = Theta2 \Comment{Aceita Theta2}

		\ElsIf{$0 \leq$ alpha $\leq 1$}
			\State f[c] = Theta2 com probabilidade alpha
			\State f[c] = Theta1 com probabiliade (1-alpha)
		\EndIf
		\State Theta1 = f[c]

		\EndWhile
		\State
		\Return f
	\end{algorithmic}
\end{algorithm}
	
Para a geração, foram usadas na geração de números de acordo com a normal e com a Dirichlet a biblioteca "Scipy". Na normal, foi usado o parâmetro "cov = 0.6", que termina a matriz de covariância que empíricamente rejeita $\approx$ 30\% dos Thetas gerados. Na Dirichlet, o parâmetro usado é gerado aleatóriamente na função "main", assim como no EP04. 

Estão abaixo o resultado de alguns testes com o gerador modificado: 

	\begin{table}[htbp]
		\begin{tabular}{||c|c|c|c|c||}
			N & v & U(v) & TEmpo de geração & Precisão  \\
			\hline
			a & b & c & d & e \\	
			a & b & c & d & e \\
		\end{tabular}
	\end{table}

	Para medida de precisão, foram gerados 100 amostras diferentes e medido quantas ficaram dentro do intervalo de confiança pedido, com a referência sendo um N = $10^7$ arbitráriamente grande e $k = 10^6$ suficiente para garantir a precisão dado N grande suficiente. 
	
	\section{Conclusão}

\end{document}