\documentclass{article}
\usepackage{graphicx} % Required for inserting images
\usepackage{amsmath}
\usepackage{caption}
\usepackage[leftcaption]{sidecap}
\usepackage{afterpage}
\usepackage{titling}
\usepackage{algorithm}
\usepackage{algpseudocode}

\predate{}
\postdate{}
\graphicspath{ {./imagens/} }
\usepackage{multicol}
\date{}
\title{\vspace{-3.0cm}EP 02 - MAP2212}
\author{Gustavo Nunes }

\begin{document}
	
	\maketitle
	
	\section{O Programa}
	O programa é o mesmo do EP04, porém com uma mudança no gerador de números de acordo com a distribuição Diricihlet, sem usar o método rvs da função Dirichlet da biblioteca Scipy. 
	
	A ideia para o gerador é usar o algorítmo de Metropolis-Hastings, que para o nosso caso pode ser escrito em pseudocódigo: 
\begin{algorithm}
	\caption{Algorítmo de Metropolis-Hastings para gerador de números aleatórios de acordo com a distribuição Diricihlet}\label{}
	\begin{algorithmic}
		\State Theta1 = ponto arbitrário 
		\State f = array de resultados 
		\While{$c \leq N$}
		\State Theta2 gerado de Normal Multivariada de média Theta1
		\State Alpha = Dirichlet(Theta2) / Dirichlet(Theta1)
		\If{alpha $\geq$ 1} 
			\State f[c] = Theta2 \Comment{Aceita Theta2}

		\ElsIf{$0 \leq$ alpha $\leq 1$}
			\State f[c] = Theta2 com probabilidade alpha
			\State f[c] = Theta1 com probabiliade (1-alpha)
		\EndIf
		\State Theta1 = f[c]

		\EndWhile
		\State
		\Return f
	\end{algorithmic}
\end{algorithm}
	
Para a geração, foram usadas na geração de números de acordo com a normal e com a Dirichlet a biblioteca "Scipy". Na normal, foi usado o parâmetro "cov = 0.6", que termina a matriz de covariância que empíricamente rejeita $\approx$ 30\% dos Thetas gerados. Na Dirichlet, o parâmetro usado é gerado aleatóriamente na função "main", assim como no EP04. 

Estão abaixo o resultado de alguns testes com o gerador modificado: 

	\begin{table}[htbp]
            \centering
		\begin{tabular}{||c|c|c|c|c||}
			N & v & U(v) & Tempo de geração\\
			\hline
			5000 & 0 & 0.0010 & 0 \\	
			5000 & 4 & 0.1150 & 0.0002 \\
			5000 & 8 & 0.2560 & 0.0005 \\
			5000 & 12 & 0.4030 & 0.0006 \\
			5000 & 16 & 0.5750 & 0.0009 \\
			5000 & 20 & 0.7150 & 0.0014 \\
			5000 & 24 & 0.8890 & 0.0016 \\
			5000 & 27 & 1 & 0.0.0016 \\
		\end{tabular}
	\end{table}

	Para medida de precisão, foram gerados 100 amostras diferentes e medido quantas ficaram dentro do intervalo de confiança pedido, com a referência sendo um N = $10^6$ arbitráriamente grande e $k = 10^4$ suficiente para garantir a precisão dado N grande suficiente. 
	
	\section{Conclusão}

        De acordo com os testes realizados e mostrados na tabela da seção anterior,
        os resultados obtidos são extremamente satisfatórios para o objetivo do que foi
        proposto. Para cada corte v, o algoritmo foi capaz de se aproximar do valor
        real da massa da função W(v) com precisão acima de 99.95\%.
        
        Vale lembrar que para gerar todos os arrays necessários, foi utilizado o NUSP
        13687175 como seed. Desse modo, fica mais fácil reproduzir os testes e verificar
        seus resultados.


        Não obstante, os tempos de execução de cada experimento após a geração dos números foram excelentes,
        não ultrapassando o tempo de 1 segundo. Claro que o valor máximo de corte
        para analise da função é relativamente pequeno (26.95 para ser mais exato),
        tornando o número de comparações também pequeno. 

 	O problema do gerador é que como cada número precisa ser gerado com um loop independentemente, não sendo possível
  	o uso de métodos mais rápidos de vetores do Numpy e do Scipy, o tempo de geração dos números 
  	é consideravelmente maior, oque é de ser esperado dado um algoritmo para situações em que não conseguimos gerar
   	os números diretamente com a distribuição desejada. 
    

\end{document}
